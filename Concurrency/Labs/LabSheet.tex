\documentclass[10pt,a4paper]{article}
\usepackage{fontspec}
\defaultfontfeatures{Mapping=tex-text}
\usepackage{xunicode}
\usepackage{xltxtra}
%\setmainfont{???}
\usepackage{amsmath}
\usepackage{amsfonts}
\usepackage{amssymb}
\author{Joseph Kehoe}
\title{Concurrency Labs}
\begin{document}
\section{Starting out}
Create a project on your github account. Call this project \textbf{CDDLabs}. 
\begin{itemize}
\item Clone the labs from github
\item Download your new repository onto your PC
\item Follow the instructions below to modify the files and complete the exercises.
\end{itemize}

Each folder will include a folder called \textbf{docs}.  All generated Doxygen files will be created inside this directory.
\section{Lab One - The Toolchain}
The purpose of this lab is to introduce you to the Unix (GNU Linux) command line tools. We will specifically use:
\begin{enumerate}
\item Git for document versioning and management (c++ files in our case)
\item Emacs for editing
\item g++ for compiling
\item gdb for debugging
\item Make for managing code projects
\item Doxygen for documenting code
\end{enumerate}



I will go through the following steps with you:
\subsection{Tasks}
\begin{enumerate}
\item Download the complete set of files using git to clone https://github.com/josephkehoe/CDD101
(If you are reading this then you have done this already ;-)
\item  	Find and open the file helloThreads.cpp using emacs (the graphical version of emacs is easier to use for beginners)
\item	Customise emacs to suit your tastes and examine the code to make sure you understand it
\item 	Compile the file from the command line using the command "g++ -std=c++11 -pthread helloThreads.cpp"
\item	Run the file and check the output making sure you understand what has occured
\item	Copy the file Makefile1 to Makefile and examine using emacs
\item	Reopen helloThreads.cpp and try the compile option
\item	Copy the file Makefile2 to Makefile and examine using emacs
\item	Repeat for Makefile3
\item	Create a file containing the following code


\begin{verbatim}
#include <stdio.h>
int main(void)
{
        int i;
        for(i=0;i<10;i++)
                printf("%d",i);
}
\end{verbatim}



\item	Compile using -g -O0 switches
\item	Step through code using gdb
\item	Use Doxygen to generate documentation for the helloThreads.cpp code
\end{enumerate}



\section{Lab Two - Signalling with Semaphores}
You will need the following files to complete this lab.
\begin{description}
\item[Semaphore.h] The header file for the \textit{Semaphore} class.  This file is provided for your use.  Make sure you understand how it works.
\item[Semaphore.cpp] The implementation file for the \textit{Semaphore} class.  This file is provided for your use.  Make sure you understand how it works.
\item[main.cpp] The file containing the main function.  This main function must create at least two threads where one thread signals the other using a common Semaphore.  In this lab this is where all your code will go.
\item[Makefile] This is the project file. It contains rules that tell the system how to compile the code and produce a working executable called signal.
\item[Doxyfile] This file contains the settings for the Doxygen tool. It is generated by the doxygen program when run for the first time.
\item[README] This is a text file describing the project. Every project must have one.
\end{description}

Edit the \textit{main.cpp} file so that the two functions (taskOne and taskTwo) are run in seperate threads and a semaphore is used to ensure that taskOne runs and exits before taskTwo.

All the code you produce must be properly commented using Doxygen, compile without error and be correct.
 
 \section{Lab Three - Rendezvous}
 Using the Semaphore class create a program that demonstrates the Rendezvous pattern.  As before (and always) you must also include the Doxygen settings file and a Makefile.  The makefile must now also include a “clean” rule that deletes all .o files from the project.
 
 \section{Lab Four - Mutual Exclusion}

Using the Semaphore class create a program that demonstrates Mutual Exclusion. You must  incldue the Doxygen settings file and a Makefile.  The makefile must now also include a “clean” rule that deletes all .o files from the project and a debug option that allows use of the gdb debugger.  

 \section{Lab Five - Reusable Barrier Class}

Create a reusable barrier class that employs the Semaphore Class. You must  include the Doxygen settings file and a Makefile.
The makefile must now also include a “clean” rule that deletes all .o files from the project and a debug option that allows use of the gdb debugger. Add the -Wall flag to the list of compiler flags sued by the Debug rule. 

All files should now begin with a file comment that contains the following information:
    1. Author Name
    2. Date of File Created
    3. Licence Employed
All code files will include suitable Doxygen comments.  The Makefile will also include comments explaining its purpose.
A README file must also be included.  This is a text file that describes what this solution does, who the author is and the licence employed.  It should contain instructions on how to compile the file and run it.
The main function should show the barrier in acton in such a way that it is clear that it works.
Files in this Lab
    1. Semaphore.cpp
    2. Semaphore.h
    3. Barrier.cpp
    4. Barrier.h
    5. main.cpp
    6. Doxyfile
    7. README
    8. Makefile

 \section{Lab Six - Producers and Consumers}
 
 
 Create a program that has two parts.  A producer and a consumer.

The producer generates random characters from ‘a’ to ‘z’ at random intervals (between 0 and 1 second in length). It adds these to a thread safe buffer that has a finite holding capacity of N characters. It generates a preset number of characters (determined at runtime) and when it has finished it add an ‘X’ character to the buffer and exits.
The consumer takes these letters from the buffer at random time intervals (between 0 and 1 second in length) and records how many of each letter it consumes. Once it sees an ‘X’ in the buffer it adds its character count to a central buffer and exits.
Files in this Lab
    1. Semaphore.cpp
    2. Semaphore.h
    3. SafeBuffer.cpp
    4. SafeBuffer.h
    5. main.cpp
    6. Doxyfile
    7. README
    8. Makefile
The main file should demonstrate your producer consumer implementation in action by creating a  number of consumers and producers and showing them in action.
All files must include suitable documentation. The Makefile must contain a rule (‘doc’) that runs the Doxygen program and generates the documentation.
Edit your emacs settings so that it now automatically generates headers for your code files.  e.g. see https://www.emacswiki.org/emacs/AutomaticFileHeaders

\end{document}